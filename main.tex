\documentclass[12pt,a4paper]{report}

%---------------------------------------------------------
% PAQUETES BÁSICOS
%---------------------------------------------------------
\usepackage[utf8]{inputenc}
\usepackage[T1]{fontenc}
\usepackage[spanish, es-nodecimal-separator]{babel}
\usepackage{setspace}
\usepackage{graphicx}
\usepackage{titlesec}
\usepackage{geometry}
\usepackage{hyperref}
\usepackage{csquotes}

%---------------------------------------------------------
% FORMATO UPCH – MÁRGENES OFICIALES
% Margen izquierdo: 3.5 cm
% Margen derecho : 2.5 cm
% Margen superior: 3.0 cm
% Margen inferior: 2.5 cm
%---------------------------------------------------------
\geometry{
    a4paper,
    left=3.5cm,
    right=2.5cm,
    top=3.0cm,
    bottom=2.5cm
}

% Interlineado 1.5
\onehalfspacing

% Títulos tamaño 14 pt
\titleformat{\chapter}{\normalfont\huge\bfseries}{\thechapter}{1em}{}
\titleformat{\section}{\normalfont\Large\bfseries}{\thesection}{1em}{}

%---------------------------------------------------------
% INICIO DEL DOCUMENTO
%---------------------------------------------------------
\begin{document}

%---------------------------------------------------------
% PORTADA OFICIAL FACIEN – UPCH
%---------------------------------------------------------
\begin{titlepage}
\centering

% --- Encabezado institucional ---
{\Large \textbf{Universidad Peruana Cayetano Heredia}}\\[0.4cm]
{\large \textbf{Facultad de Ciencias y Filosofía “Alberto Cazorla Talleri”}}\\[1.8cm]

% --- Logo UPCH (colocar logo-upch.png en la carpeta) ---
\includegraphics[width=7cm]{logo upch.png}\\[3.0cm]

% --- Título de la tesis ---
{\LARGE \bfseries TÍTULO DE LA TESIS}\\[2.2cm]

% --- Autor ---
{\large Tesis presentada por}\\[0.4cm]
{\Large \textbf{Nombre Completo del Autor}}\\[1.4cm]

% --- Título profesional ---
{\large Para optar el título profesional de}\\[0.4cm]
{\Large \textbf{Licenciado en …}}\\[2.2cm]

% --- Asesor ---
{\large Asesor: Dr./Mg. Nombre del Asesor}\\[2.2cm]

% --- Pie de página ---
{\large Lima, Perú}\\[0.2cm]
{\large Año}

\end{titlepage}

% Numeración romana para preliminares
\pagenumbering{roman}

%---------------------------------------------------------
% REPORTE TURNITIN (OBLIGATORIO)
%---------------------------------------------------------
\chapter*{Reporte de similitud Turnitin}
\addcontentsline{toc}{chapter}{Reporte de similitud Turnitin}
% Insertar imagen si deseas:
% \includegraphics[width=\textwidth]{turnitin.png}

%---------------------------------------------------------
% ÍNDICE
%---------------------------------------------------------
\tableofcontents
\cleardoublepage

% Cambio a numeración arábiga
\pagenumbering{arabic}

%---------------------------------------------------------
% RESUMEN EN ESPAÑOL
%---------------------------------------------------------
\chapter*{Resumen}
\addcontentsline{toc}{chapter}{Resumen}

Escribe aquí el resumen en español. Máximo una página.

\textbf{Palabras clave:} palabra1, palabra2, palabra3.

%---------------------------------------------------------
% ABSTRACT EN INGLÉS
%---------------------------------------------------------
\chapter*{Abstract}
\addcontentsline{toc}{chapter}{Abstract}

Write here the abstract in English. Maximum one page.

\textbf{Keywords:} keyword1, keyword2, keyword3.

%---------------------------------------------------------
% 1. INTRODUCCIÓN
%---------------------------------------------------------
\chapter{Introducción}

Describe el problema, marco teórico, antecedentes y propósito del estudio.

%---------------------------------------------------------
% 2. HIPÓTESIS / PREGUNTA DE INVESTIGACIÓN
%---------------------------------------------------------
\chapter{Hipótesis / Pregunta de Investigación}

Escribe aquí la hipótesis central o la pregunta de investigación.

%---------------------------------------------------------
% 3. OBJETIVOS
%---------------------------------------------------------
\chapter{Objetivos}

\section{Objetivo general}
Escribe el objetivo general.

\section{Objetivos específicos}
\begin{enumerate}
    \item Objetivo específico 1.
    \item Objetivo específico 2.
    \item Objetivo específico 3.
\end{enumerate}

%---------------------------------------------------------
% 4. MATERIALES Y MÉTODOS
%---------------------------------------------------------
\chapter{Materiales y Métodos}

Describe materiales, equipos, muestras, diseño,
protocolos, análisis estadístico y, si corresponde, tu diagrama de Gantt.

%---------------------------------------------------------
% 5. RESULTADOS
%---------------------------------------------------------
\chapter{Resultados}

Presenta texto + figuras + tablas.

Ejemplo:

\begin{figure}[h]
\centering
\includegraphics[width=0.7\textwidth]{123.png}
\caption{Leyenda explicativa de la figura.}
\end{figure}

%---------------------------------------------------------
% 6. DISCUSIÓN
%---------------------------------------------------------
\chapter{Discusión}

Relaciona tus resultados con literatura, explica implicancias,
limitaciones y aportes.

%---------------------------------------------------------
% 7. CONCLUSIONES
%---------------------------------------------------------
\chapter{Conclusiones}

Redacta conclusiones claras, derivadas de los resultados.

%---------------------------------------------------------
% 8. REFERENCIAS BIBLIOGRÁFICAS
%---------------------------------------------------------
\chapter{Referencias Bibliográficas}

\bibliographystyle{vancouver}
\bibliography{bibliografia}

%---------------------------------------------------------
% ANEXOS
%---------------------------------------------------------
\appendix
\chapter{Anexos}

Incluye mapas, tablas extendidas, protocolos, etc.

\end{document}
